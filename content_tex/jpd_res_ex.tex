%-------------------------------------------------------------------------------
%	SECTION TITLE
%-------------------------------------------------------------------------------
\vspace{-0.8cm}
\cvsection{Research Experience}

%-------------------------------------------------------------------------------
%	CONTENT
%-------------------------------------------------------------------------------
\begin{cventries}

%---------------------------------------------------------
%---------------------------------------------------------
\vspace{-0.2cm}
\cventry
    {Supervisor: \textbf{Dr. Tanvir Hossain}} % Position
    {1. Undergraduate Research Assistant} % Organization / Project name
    %{\textcolor{gray}{ Advance Bioinformatics Laboratory, BMB, SUST}}
    {\textbf{\textcolor{gray}{BMB, SUST}}} % Location
    {\textbf{\textcolor{gray}{Oct 2022 -- Jul 2024}}} % Dates
    %{\textbf{\fontseries{sb}\selectfont Feb 2020 -- Aug 2025}}
    {%
    \begin{cvitems}
      \item[$\blacksquare$] \textbf{ Responsibilities and Experience}
        \begin{itemize}
          \color{black}
          \item[\textbf{--}] Implemented a full RNA-seq workflow in bash for gene quantification from GEO/SRA FASTQ files and applied statistical models such as edgeR, limma, and DESeq2 to get DEGs.
          \item[\textbf{--}] Constructed collective dynamics of gene networks to reveal functional modules and hub genes enriched in biological pathways. 
          \item[\textbf{--}] Benchmarked fifteen supervised learning models in PyCaret and utilized the best-fitted classifier to validate the biomarkers based on discriminatory power.
          \item[\textbf{--}] Constructed a multi-source gene expression database with DE matrices and used MLPRegressor for training DeepFoldChange.
        \end{itemize}
      \item[$\blacksquare$] \textbf{ GitHub @ } \href{https://github.com/Prokash21/2022_MPXV_Project}{\textcolor{cvblue}{Mpox-Project}}, \href{https://github.com/Prokash21/biomarker-discovery}{\textcolor{cvblue}{Biomarker-Discovery}}, \href{https://github.com/Prokash21/RNA-Seq-Analysis}{\textcolor{cvblue}{Bulk-RNA-seq}},
      \href{https://github.com/Prokash21/DeepFoldChange}{\textcolor{cvblue}{DeepFoldChange}}
    \end{cvitems}
    }

% \item[$\blacksquare$] \faGithub \ 
\vspace{0.35cm}
\cventry
    {Supervisor: \textbf{Dr. Ajit Ghosh}} % Position
    {2. Undergraduate Research Assistant} % Organization / Project name
    {\textbf{\textcolor{gray}{BMB, SUST}}} % Location
    %{Nov 2023 -- Oct 2025} % Dates
    {\textbf{\fontseries{sb}\selectfont Nov 2023 -- Oct 2025}}
    {%
    \begin{cvitems}
      \item[$\blacksquare$] \textbf{ Responsibilities and Experience}
        \begin{itemize}
          \color{black}
          \item[\textbf{--}] Extracted RNA from plant samples, quantified by nanodrop, and performed PCR with EF1α (housekeeping) and target-gene primers to confirm cDNA synthesis and primer specificity simultaneously, and visualized bands on agarose gel electrophoresis.
          \item[\textbf{--}] Validated candidate biomarkers by qRT-PCR, calculating relative log2 fold change from Ct values via the ΔΔCt method.
          \item[\textbf{--}] Profiled conserved domains and motifs of m6A regulators; built phylogenies with 1000-bootstrap in MEGA, and visualized in iTOL.
        \end{itemize}
      \item[$\blacksquare$] \textbf{ GitHub @ } \href{https://github.com/Prokash21/BioSalT}{\textcolor{cvblue}{BioSalT}},
      \href{https://github.com/Prokash21/Genome-Wide}{\textcolor{cvblue}{Genome-Wide}} 
    \end{cvitems}
    }

\vspace{0.2cm}
\cventry
  %{Supervisor: Preonath Chondrow Dev} % Position
  {Supervisor: \textbf{Preonath Chondrow Dev}}
  {3. Research Assistant} % Organization / Project name
  %{Child Health Research Foundation (CHRF), Dhaka} % Location
  {\textbf{\textcolor{gray}{CHRF, Dhaka}}}
  %{Jun 2024 – Present} % Dates
  {\textbf{\textcolor{gray}{Jun 2024 – Present}}}
  {%
    \begin{cvitems}
      \item[$\blacksquare$] \textbf{ Responsibilities and Experience}
        \begin{itemize}
          \color{black}
          \item[\textbf{--}] Developed omicML (GUI) for biologist (non-programmers) to build biomarker algorithms using transcriptomic data.
          \item[\textbf{--}]Developing a computational model integrating electrophysiology and transcriptomics data for epilepsy (from the Allen Brain Atlas) to explain how gene expression patterns relate to the electrical properties of neurons.
          \item[\textbf{--}]Analyzing brain MRI and surface based brain morphometry data to study the structure and function of the brain.
        \end{itemize}
      \item[$\blacksquare$] \textbf{ GitHub @ } \href{https://github.com/Prokash21/omicML_raw}{\textcolor{cvblue}{omicML-raw}}, \href{https://github.com/Prokash21/Celltypes-Patchseq-Epilepsy}{\textcolor{cvblue}{Celltypes-Patchseq}}, \href{https://github.com/Prokash21/Neuroimaging}{\textcolor{cvblue}{Neuroimaging}}, \href{https://github.com/Prokash21/Allen-Brain}{\textcolor{cvblue}{Allen-Brain}}
    \end{cvitems}
  }

\vspace{0.4cm}
\cventry
  %{Advisor: Dr. Tanvir Hossain} % Position
  {Advisor: \textbf{Dr. Tanvir Hossain}}
  {4. Remote Research Intern} % Organization / Project name
  {\textbf{\textcolor{gray}{AIBN, UQ, Australia}}}
  %{Australian Institute for Bioengineering and Nanotechnology - AIBN, UQ, Australia} % Location
  {\textbf{\textcolor{gray}{Feb 2025 – Aug 2025}}}
  %{Feb 2025 – Aug 2025} % Dates
  {%
    \begin{cvitems}
      \item[$\blacksquare$] \textbf{ Responsibilities and Experience}
        \begin{itemize}
          \color{black}
          \item[\textbf{--}] Conducted bulk ncRNA-seq analysis on cancer cell lines, including ADMSC, BMMSC, HeLa, MCF7, MDAMB231, TM6, A549, H1975.
          \item[\textbf{--}] Analyzed the expression profiles of Y and U glycoRNAs to examine their significance in extracellular vesicles (EVs), epithelial–mesenchymal transition (EMT) and in lung cancer. 
          \item[\textbf{--}] Filtered ncRNAs utilizing ncRNAtools part of RNAcentral API and selected best features by RFE-RF for ML analysis.
          \item[\textbf{--}] Discovered nc-markers of EMT, EV, and lung cancer and common among all and analyzed their qRT-PCR validation result.
        \end{itemize}
      \item[$\blacksquare$] \textbf{ Project}
        \begin{itemize}
          \item[\textbf{--}] \textbf{Chip development for efficient glycoRNA isolation and marker-based cancer detection.} \textbf{GitHub} @ [\href{https://github.com/Prokash21/glycoRNA-UQ-Australia}{\textcolor{cvblue}{glycoRNA}}]
        \end{itemize}
    \end{cvitems}
    }

\vspace{0.4cm}
\cventry
  {Supervisor: \textbf{Dr. S M Rashed Ul Islam}}
  %{Supervisor: Dr. S M Rashed Ul Islam} % Position
  {5. Research Assistant} % Organization / Project name
  {\textbf{\textcolor{gray}{BMU, Dhaka}}}
  %{Bangladesh Medical University (BMU), Dhaka} % Location
  {\textbf{\textcolor{gray}{Nov 2024 – Present}}}
  %{Nov 2024 – Present} % Dates
  {%
    \begin{cvitems}
      \item[$\blacksquare$] \textbf{ Responsibilities and Experience}
        \begin{itemize}
          \color{black}
          \item[\textbf{--}] Detected malignant samples of HNSCC by hispopathology, followed by multiplex and nested PCR to screen HPV.
          \item[\textbf{--}] Utilized sanger sequencing of L1 viral gene and immunohistochemistry of host proteins (upregulated by HPV) for further validation. 
          \item[\textbf{--}] Conducting molecular techniques to validate cancer-specific biomarkers retrieved through a multi-omics approach, including WGCNA, scRNA-seq, and proteomics.
        \end{itemize}
      \item[$\blacksquare$] \textbf{ Onging Project}
      \vspace{0.08cm}
        \begin{itemize}
          \item[\textbf{--}] \textbf{Machine learning-driven identification and quantitative validation of cancer-specific biomarkers across multiple carcinomas of different anatomic sites}, Ongoing Research. \href{https://github.com/Prokash21/Deep-Neural-Profiling/blob/main/BMU_PROJECT/README_BMU_PROJECT.md}[{\textcolor{cvblue}{Collaborators}}]. \textbf{GitHub} @ \href{https://github.com/Prokash21/scRNA-seq}{\textcolor{cvblue}{scRNA-seq}}
        \end{itemize}
    \end{cvitems}
    }
    
\vspace{0.4cm}
\cventry
  {Supervisor: \textbf{Papia Rahman}}
  %{Supervisor: Papia Rahman} % Position
  {6. Research Assistant} % Organization / Project name
  {\textbf{\textcolor{gray}{BUET; BMU; SUST}}}
  %{Department of Chemistry, BUET · BMU · SUST} % Location
  {\textbf{\textcolor{gray}{Mar 2025 – Present}}}
  %{Mar 2025 – Present} % Dates
  {%
    \begin{cvitems}
      \item[$\blacksquare$] \textbf{ Responsibilities and Experience}
        \begin{itemize}
          \color{black}
          \item[\textbf{--}] Utilized DeepProfile framework for Oropharyngeal Carcinoma, integrating 26 GEO datasets and reduced the dimension with PCA for Variational Autoencoder (VAE) model training.
          \item[\textbf{--}] Streamlined Integrated Gradients (IG), ensemble latent feature learning and TCGA survival benchmarking.
          \item[\textbf{--}] Synthesized Ag‑deposited Ni/rGO nanoparticles and evaluated their peroxidase mimicking, glucose‑sensing efficiency, antibacterial, and antioxidant properties.
        \end{itemize}
      \item[$\blacksquare$] \textbf{ Ongoing Collaborative Research}
        \begin{itemize}
          \item[\textbf{--}] \textbf{Synthesis and application of Ag-deposited Ni/rGO nanospheres in glucose sensing, antibacterial properties, and antioxidant activity}, Funded by SUST Research Center. \textbf{GitHub} @ \href{https://github.com/Prokash21/Deep-Neural-Profiling}{\textcolor{cvblue}{Deep-Neural-Profiling}}
        \end{itemize}
    \end{cvitems}
    }
  
\end{cventries}